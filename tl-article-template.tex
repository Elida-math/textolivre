\documentclass[red]{textolivre}
\usepackage[utf8]{inputenc}
\usepackage[portuguese,english]{babel}

\usepackage{amsmath}
\usepackage{amsfonts}
\usepackage{amssymb}
\usepackage{amsthm}

\usepackage{csquotes}

\usepackage{graphicx} 
\usepackage{booktabs}

% used to create dummy text for the template file
\usepackage{lipsum} 

\usepackage{setspace} % for \onehalfspacing and \singlespacing macros
\onehalfspacing 

\def \tituloportugues {Modelo de artigo para a revista Texto Livre}
\def \tituloingles {Article template for Free Text Journal}
\title{%
    \vspace{-15mm}%
    \selectfont\textbf{\MakeUppercase{\tituloportugues}} \\
    \vspace{2.1mm}
    \selectfont\textbf{\textit{{\MakeUppercase{\tituloingles}}}}%
} 

\date{Dezembro de 2019}

%\author{Leonardo Carneiro de Araujo \authorcr 
%\small Universidade Federal de São João del Rei - Brasil \authorcr 
%\small \email{leolca@ufsj.edu.br} \authorcr \vskip1.3cm
%Daniervelin Renata Marques Pereira \authorcr 
%\small Universidade Federal de Minas Gerais - Brasil \authorcr
%\small \email{benevides.aline12@gmail.com} \authorcr 
%}
%\author[1]{Leonardo Carneiro de Araujo}
%\affil[1]{Universidade Federal de São João del Rei, Brasil}
%\author[2]{Daniervelin Renata Marques Pereira}
%\affil[2]{Universidade Federal de Minas Gerais, Brasil}
%%% Authors should be input with the following commands
\tlauthor{Leonardo Carneiro de Araujo}{Universidade Federal de São João del Rei, Brasil}{leolca@ufsj.edu.br}
\tlauthor{Daniervelin Renata Marques Pereira}{Universidade Federal de Minas Gerais, Brasil}{revista@textolivre.org}


\begin{otherlanguage}{english}
\begin{abstract}
% write your abstract here
This article is a template article which aims to guide authors who will submit their papers to the Texto Livre journal. This template should facilitate their endeavor, viewing and configuring the text in the format already configured accordingly. Just write or paste your content in the desired places, replacing texts and figures, checking if they keep up the format. The abstract should have between 150 and 200 words.
\keywords{word1; word2; many-words; word3}
\end{abstract}
\end{otherlanguage}

\begin{resumo}
% escreva o resumo aqui
Este artigo é um modelo que visa orientar os autores que submeterão seus textos à revista Texto Livre a fazê-lo no padrão determinado pela revista a fim de facilitar sua empreitada, visualizando e configurando o texto no formato já configurado nesse modelo. Para fazê-lo, basta escrever ou colar seus textos no lugar aqui designado, substituindo os textos e figuras aqui existentes, verificando se eles mantêm a formatação aqui descrita. O resumo deverá ter entre 150 e 200 palavras.
\palavraschave{palavra1; palavra2; muitas-palavras; palavra3}
\end{resumo}

\usepackage[backend=biber,style=abnt, ittitles]{biblatex}
\DeclareLanguageMapping{brazil}{brazil-apa}
\addbibresource{tl-article-template.bib}     
% use biber instead of bibtex
% $ biber tl-article-template
% $ pdflatex tl-article-template.tex


\begin{document}
\selectlanguage{portuguese}
\maketitle

\section{Introdução}
\lipsum[1-10]
\lipsum[11][1-2]\cite{donaldknuth1984,leslielamport1994}\lipsum[11][3-5]

\begin{figure}[htbp]
 \centering
 \includegraphics[width=0.5\textwidth]{example-image-a}
 \caption{Imagem de teste.}
 \label{fig-img-a}
\end{figure}

\lipsum[12]

\section{Metodologia}
\lipsum[13-14]
\begin{equation}\label{eq-bin}
\binom{n}{k} = \frac{n!}{k!(n-k)!}
\end{equation}
\lipsum[15] (ver Equação \ref{eq-bin} e Tabela \ref{tab-exemplo}).\footnote{\lipsum[30]}

\begin{table}[htbp]
\centering
\caption{Tabela de exemplo para o Texto Livre}\label{tab-exemplo}
\begin{tabular}{cccc}
\toprule
  Dec  & Bin       & Octal & Hexa \\
\midrule  
  33   & 100001    &  41   & 21   \\
\midrule
  117  & 1110101   & 165   & 75   \\
\midrule
  451  & 111000011 & 703   & 1C3  \\
\midrule
  431  & 110101111 & 657   & 1AF  \\
\bottomrule
\end{tabular}
\end{table}

\lipsum[16]

\begin{equation}\label{eq-plancherel}
\int_{-\infty}^\infty \left| f(x) \right|^2\,dx = \int_{-\infty}^\infty \left| \hat{f}(\xi) \right|^2\,d\xi.
\end{equation}

\lipsum[20-21]

\section{Conclusão}
\lipsum[17-19]
Equação \ref{eq-plancherel} e Tabela \ref{tab-exemplo}.
\lipsum[18]


% Referências
\printbibliography[title=Refer\^{e}ncias]

\end{document}

